\section{Learning Algorithm}
\label{sec:learning}

Here we present our algorithm, \alg,
for learning symbolic automata.
The premise is that the automaton to be learned,
called the \emph{target},
is hidden in a black box,
so knowledge of it comes from some \emph{oracle}
that admits two kinds of queries:
\emph{membership queries} that ask
whether a word is in the language of the target, and
\emph{equivalence queries} that ask
whether a conjectured automaton is equivalent
to the target---if not, a 
counterexample is provided.
\alg, which builds upon L$^*$~\cite{angluin87},
maintains an \emph{observation table}
that comprises its knowledge about the target.
The observation table is used to build the intermediary guesses of the target automaton
and, eventually, the final automaton.
It is assumed that the learner knows
both the alphabet and the Boolean algebra in question. 
%We assume the learner knows the Boolean algebra over which that automaton operates.

\subsection{Observation Table}

The observation table consists of
rows of prefixes and columns of suffixes.
Each entry determines whether the target automaton
accepts the word formed by concatenating
the prefix and suffix.
Intuitively, prefixes
provide knowledge about words that
lead to states,
and suffixes help differentiate those states.

\begin{definition}[Observation Table]
    An \emph{observation table} $T$ for an \SFA\ $M$
    is a tuple $(\Sigma, S, R, E, f)$ where
    $\Sigma$ is a potentially infinite set called the \emph{alphabet};
    $S,R,E \subset \Sigma^*$ are finite subsets of words:
        $S$ is called the set of \emph{prefixes}, 
        $R$ is called the \emph{boundary},
        and $E$ is called the set of \emph{suffixes};
    $f : (S \cup R) \cdot E \rightarrow \{0,1\}$
        is a \emph{classification function}\footnote{We
        also use $\{-,+\}$ to denote the range of $f$.}
        such that
        for a word $w \cdot e \in (S \cup R) \cdot E$,
        $f(w \cdot e) = 1$ if $w \cdot e \in \mathcal{L}(M)$, and
        $f(w \cdot e) = 0$ if $w \cdot e \not\in \mathcal{L}(M)$.%
        \footnote{We use $\cdot$ to denote both the concatenation
        of strings and its lifting to sets of strings,
        as is standard.}
    Additionally, %the following properties hold:
    \rone $S$ and $R$ are disjoint,
    \rtwo $S \cup R$ is prefix-closed and $\epsilon \in S$,
    \rthree for all $s \in S$, there exists a character $a \in \Sigma$ such that $s \cdot a \in R$,
    and \rfour $\epsilon \in E$.
\end{definition}
%
Table $\mathbf{T_1}$ in Figure~\ref{fig:example} is an example observation table:
The rows begin with elements of $S \cup R$,
where the elements in $S$ are shown above the horizontal divider
and the elements in $R$ below,
and the columns begin with elements of $E$.

The observation table induces the construction of an automaton.
For intuition,
each $s \in S$ corresponds to a state $q$ 
such that $s$ is a string of moves
from $q_\text{init}$ to $q$.
The boundary $R$ gives information about the 
transitions between states.
The states are differentiated by the 
strings in $E$ and the classification function $f$,
as if there exist $s_1, s_2 \in S$ and $e \in E$
such that $f(s_1 \cdot e) \neq f(s_2 \cdot e)$,
then $s_1$ and $s_2$ \emph{behave} differently 
and must lead to different states.
We use the notation $\textit{row}(w)$ for $w \in S \cup R$
to denote the vector indexed by $e \in E$ of $f(w \cdot e)$.

\alg\ manipulates the observation table and eventually
conjectures an \SFA.  For this to happen, the table must first satisfy certain properties---we
call such a table \emph{cohesive}---that
are established through membership queries to the oracle.
The cohesive observation table is used
to construct an intermediary automaton
that is ultimately used to produce a conjectured \SFA.
An observation table is \emph{closed} 
if for each $r \in R$ there exists $s \in S$
such that $\textit{row}(s) = \textit{row}(r)$;
in other words, each element in the boundary
corresponds to a state.
%
An observation table is \emph{reduced}
if for all $s_1, s_2 \in S$,
$\textit{row}(s_1) \neq \textit{row}(s_2)$,
meaning each state is uniquely characterized by $f$ and $E$.
%
An observation table is \emph{consistent}
if for all $w_1, w_2 \in S \cup R$,
if $a \in \Sigma^*$ and $w_1 \cdot a, w_2 \cdot a \in S \cup R$
and $\textit{row}(w_1) = \textit{row}(w_2)$, then
$\textit{row}(w_1 \cdot a) = \textit{row}(w_2 \cdot a)$.
%
A table being consistent means that if the words $w_1$ and $w_2$
are equivalent according to $f$ and $E$,
then $w_1 \cdot a$ and $w_2 \cdot a$ ought to be
equivalent as well, and thus
there is no evidence to the contrary.\footnote{We
use $a \in \Sigma^*$ for the definition
of \emph{consistent}, but since the table
is prefix-closed by definition,
it is equivalent to consider
only single-characters $a \in \Sigma$.}
%
An observation table is \emph{evidence-closed}
if for all $e \in E$ and $s \in S$,
$s \cdot e \in S \cup R$.
%
An observation table is \emph{cohesive}
if it is closed, reduced, consistent, and evidence-closed.

Consider, for example, the observation tables in Figure~\ref{fig:example}.
$\mathbf{T_2}$  is not closed, since $\textit{row}(51) = -$
and there is no $s \in S$ with $\textit{row}(s) = -$.
Table $\mathbf{T_5}$ is not
consistent because $\textit{row}(51) = - = \textit{row}(51,0)$,
but $\textit{row}(51\cdot0) = - \neq + = \textit{row}(51,0\cdot0)$.
Table $\mathbf{T_{11}}$ is closed, reduced, consistent, and evidence-closed.

If an observation table is cohesive,
then it admits the construction of an \emph{evidence automaton}
that classifies words $w \in \Sigma^*$
equivalently to the observation table's classification function $f$.

\begin{definition}[Evidence Automaton]
    An evidence automaton is a tuple
    $(\Sigma, Q, q_\text{init}, F, \Delta)$ where
    $\Sigma$ is a set;
    $Q$ is a finite set of states;
    $q_\text{init} \in Q$ is the \emph{initial state};
    $F \subseteq Q$ is the set of \emph{final states};
    $\Delta \subseteq Q \times \Sigma \times Q$ is the \emph{transition relation}.
\end{definition}
A move $\rho = (q_1, a, q_2)$,
also denoted $q_1 \xrightarrow{a} q_2$,
is a transition from $q_1$ to $q_2$ using the character $a$.  
A word $w = a_1 a_2 \ldots a_k$ is \emph{accepted at state $q$}
if for $1 \leq i \leq k$ there exist moves
$q_{i-1} \xrightarrow{a_i} q_i$ such that $q_0 = q$ and $q_k \in F$.
Conversely, if that $q_k \not \in F$, 
then $w$ is \emph{not} accepted at $q$.
If there is no path through the automaton for $w$,
then the acceptance is undefined.
An evidence automaton differs from an \SFA
in that transitions carry characters in $\Sigma$ instead of predicates
in a Boolean algebra over the domain $\Sigma$.
Additionally, the evidence automaton can be deliberately sparse:
it is not \emph{complete}, and we avoid 
the notion that a state $q$ does not accept a character $a$
if there is no $q'$ such that $(q, a, q') \in \Delta$---as
stated above, such
a case simply indicates the behavior of $a$ at $q$ is undefined.

Given a cohesive observation table
$T = (\Sigma, S, R, E, f)$, we build an evidence automaton
$A = (\Sigma, Q, q_\text{init}, F, \Delta)$
%over that same alphabet
as follows:
for each $s \in S$, we introduce a state $q_s \in Q$. %, and
$q_\text{init}$ is assigned to $q_\epsilon$.
The final state set $F$ contains   all $q_s$ such that $s \in S$ and $f(s) = 1$.
%
Since the observation table is closed and reduced,
there exists a function $g : S \cup R \rightarrow S$
such that $g(w) = s$ if and only if $row(w) = row(s)$.
This function allows us to define the transition relation of $A$:
if $w \cdot a \in S \cup R$ for $w \in \Sigma^*$ and $a \in \Sigma$, then 
$(q_{g(w)}, a, q_{g(w \cdot a)}) \in \Delta$.
%
In Figure~\ref{fig:example}, the automaton $\mathbf{M_{1}^e}$ (resp $\mathbf{M_{11}^e}$) is the evidence automaton 
corresponding to cohesive table $\mathbf{T_{1}}$ (resp. $\mathbf{T_{11}}$).


\begin{lemma}[Evidence compatibility]\label{thm:evid}
    Given a cohesive observation table $T = (\Sigma, S, R, E, f)$,
    if $M_\text{evid} = (\Sigma, Q, q_\text{init}, F, \Delta)$ 
    is the evidence automaton construction of $T$,
    then for all $w \cdot e \in (S \cup R) \cdot E$,
    if $f(w \cdot e) = 1$ then $M_\text{evid}$ accepts $w \cdot e$, and
    if $f(w \cdot e) = 0$ then $M_\text{evid}$ does not accept $w \cdot e$.
\end{lemma}

\subsection{Separating Predicates}

Given an evidence automaton with an alphabet $\Sigma$,
we require two pieces to build an \SFA:
\rone a Boolean algebra $\mathcal{A}$ with $\mathfrak{D}_\mathcal{A} = \Sigma$,
and \rtwo a partitioning function $P$ for $\mathcal{A}$, which we define below.
%
This latter component,
the partitioning function,
is the key insight to \alg's
generalization of L$^*$.

\begin{definition}[Partitioning function]
    A \emph{partitioning function} for a Boolean algebra
    $\mathcal{A} = (\mathfrak{D}, \Psi, \den{\_}, \bot, \top, \vee, \wedge, \neg)$
    is a function $P : (2^\mathfrak{D})^* \rightarrow \Psi^*$
    that takes as input a list $L_\mathfrak{D} = \ell_1 \ldots \ell_k$
    of disjoint sets of elements in $\mathfrak{D}$,
    and returns a parallel list $L_\Psi = \varphi_1 \ldots \varphi_k$ of predicates in $\Psi$ such that
    \begin{itemize}
        \item $\bigvee_{\varphi_i \in L_\Psi} \varphi_i = \top$
        \item $\varphi_i \wedge \varphi_j = \bot$ for all $\varphi_i, \varphi_j \in L_\Psi$ with $i \neq j$
        \item for each $\ell_i \in L_\mathfrak{D}$ corresponding to $\varphi_i \in L_\Psi$,
            all $a \in \ell_i$ are such that $a \in \den{\varphi_i}$.
    \end{itemize}
\end{definition}

\begin{example}[Equality Algebra Separating Predicates]
    We can construct a partitioning function for the equality algebra:
    given a list
    $L_\mathfrak{D} = \ell_1 \ldots \ell_k$
    we construct a list $L_\Psi = \varphi_1 \ldots \varphi_k$
    where each $\varphi_i$ has $\den{\varphi_i} = \ell_i$.
    We choose a $j$ with maximal $|\ell_j|$
    and update $\varphi_j \gets \varphi_j \vee \bigwedge_{1 \leq i \leq k} \neg \varphi_i$.
    In the concrete case of $\mathfrak{D} = \mathbb{Z}$ 
    and $L_\mathfrak{D} = [\{2\}, \{3\}, \emptyset, \{0, 5\}]$,
    the partitioning function would produce (after simplification)
    %$L_\Psi = [a = 2, a = 3, \bot, a \neq 2 \wedge a \neq 3]$.
    $L_\Psi = [\lambda a \ldotp a = 2, \lambda a \ldotp a = 3, \bot, \lambda a \ldotp a \neq 2 \wedge a \neq 3]$.
\end{example}

At a high level, as long as the \SFA is consistent with the evidence automaton,
it will be consistent with the observation table.
The words in the remainder of $\Sigma^*$---for which 
the evidence automaton has unspecified classification---can be
assigned to paths in a largely arbitrary manner.
The partitioning function handles generalizing the concrete evidence
by creating \emph{separating predicates}, 
in effect specifying the behavior for the remaining words.
Ideally, this generalization allows an automaton to be learned
with a relatively small observation table,
even if the alphabet is large---or even infinite.

Given an evidence automaton $A = (\Sigma, Q, q_\text{init}, F, \Delta)$,
a Boolean algebra $\mathcal{A}$ with domain $\Sigma$, and an 
appropriate partitioning function $P$,
we build an \SFA $M = (\mathcal{A}, Q, q_\text{init}, F, \Delta_M)$
using that Boolean algebra and that exact configuration of states.
All that remains is the construction of the transition relation $\Delta_M$.

For each  $q \in Q$, we perform the following.
We gather all evidence transitions out of $q$
into a set $\Delta_q = \{(q, a, q') \in \Delta\}$
and construct a list $L_\Sigma$ indexed over the states $q_i \in Q$,
where each set in $L_\Sigma$ is $\ell_i = \{a \mid (q, a, q_i) \in \Delta_q\}$.
We apply the partitioning function to 
get a list of separating predicates
$L_{\Psi_\mathcal{A}} = P(L_\Sigma)$
which is also indexed over $q_i \in Q$,
and add $(q, \varphi_i, q_i)$ to $\Delta_M$ for each $\varphi_i \in L_{\Psi_\mathcal{A}}$.

\begin{lemma}[\SFA evidence compatibility]\label{thm:evid2}
    Given a cohesive observation table $T = (\Sigma, S, R, E, F)$,
    if $M_\text{evid} = (\Sigma, Q, q_\text{init}, F, \Delta)$
    is the evidence automaton construction of $T$, and
    $M = (\mathcal{A}, Q, q_\text{init}, F, \Delta)$
    is produced from $M_\text{evid}$ using a partitioning function,
    then for all $w \cdot e \in (S \cup R) \cdot E$,
    if $f(w \cdot e) = 1$ then $w \cdot e \in \mathcal{L}(M)$, and
    if $f(w \cdot e) = 0$ then $w \cdot e \not \in \mathcal{L}(M)$.
\end{lemma}

An example observation table, its corresponding evidence automaton, 
and a resultant \SFA are shown in the last row of Figure~\ref{fig:example}.



\subsection{Algorithm Description}

We now present a description of \alg and an example execution.
The algorithm begins by initializing an observation table
with $S = \{\epsilon\}$, $R = \{a\}$ for an arbitrary $a \in \Sigma$,
and $E = \{\epsilon\}$.
$f$ is initially undefined.
The knowledge of the table
is grown using the operations
\emph{fill}, \emph{close}, \emph{evidence-close}, and \emph{make-consistent}.

The operation \emph{fill} asks a membership query
for all $w \cdot e \in (S \cup R) \cdot E$ for which
$f$ is undefined and then adds those results to $f$;
in this way, it ensures $f$ is defined over the 
entire domain of the observation table.

The operation \emph{close} checks for
the existence of an $r \in R$ such that
for all $s \in S$, $\textit{row}(r) \neq \textit{row}(s)$.
Such an $r$ is moved from $R$ to $S$,
and $r \cdot a$ is added to $R$
for some arbitrary $a \in \Sigma$.

The operation \emph{evidence-close} ensures
for all $e \in E$ and $s \in S$ that
$s \cdot e \in S \cup R$
by adding to $R$ all $s \cdot e$
that are not.
It also adds to $R$ any necessary prefixes
so that $S \cup R$ is prefix-closed.

The operation \emph{make-consistent} operates as follows:
if there exist $w_1, w_2 \in S \cup R$
and $w_1 \cdot a, w_2 \cdot a \in S \cup R$
for some $a \in \Sigma$
such that $\textit{row}(w_1) = \textit{row}(w_2)$
but $\textit{row}(w_1 \cdot a) \neq \textit{row}(w_2 \cdot a)$,
then $w_1$ and $w_2$ actually lead to different states;
using the $e \in E$ such that $f(w_1 \cdot a \cdot e) \neq f(w_2 \cdot a \cdot e)$,
it is clear $a \cdot e$ thus differentiates those states.
Accordingly, $a \cdot e$ is added to $E$.
Additionally, we then add
$(\{u_2 \cdot b \mid u_1 \cdot b \in S \cup R\} \cup \{u_1 \cdot b \mid u_2 \cdot b \in S \cup R\}) \setminus S$
to $R$ for all pairs $u_1, u_2 \in S \cup R$ such that before adding $e$ to $E$,
$\textit{row}(u_1) = \textit{row}(u_2)$,
but $f(u_1 \cdot e) \neq f(u_2 \cdot e)$
(this includes the pair $w_1, w_2$).
This operation \emph{distributes} the old evidence leading
    out of the amalgamated state between the newly differentiated states,
simplifying the constructions in Section~\ref{sec:learnability}.

Upon receiving a counterexample $c \in \Sigma^*$ from 
an equivalence query sent to the oracle,
all prefixes of $c$ are added to $R$
(except those already present in $S$).
There are two cases for a counterexample:
one of the predicates in the \SFA needs refinement,
which is facilitated by adding those new words to the table,
or a new state must exist in the automaton,
which is handled by \emph{make-consistent}.

\algrenewcommand\algorithmicindent{1.0em}
\newcommand\NoThen{\renewcommand\algorithmicthen{}}
\algtext*{EndWhile}% Remove "end while" text
\algtext*{EndIf}
\algtext*{EndProcedure}
\begin{figure}[!t]
%\begin{minipage}{0.5\textwidth}
%%\begin{algorithm}[t]
%%    \caption{Learning algorithm}
%    \scriptsize
%    \begin{algorithmic}
%       \Procedure{\alg}{$M$}
%        \State initialize table $T$
%        \Repeat
%        \While{$T$ is not cohesive}
%        \State \emph{make-consistent} $T$
%        \State \emph{evidence-close} $T$
%        \State \emph{close} $T$
%        \EndWhile
%        \State build evidence automaton $M_\textit{evid}$ from $T$
%        \State build \SFA $M$ from $M_\textit{evid}$
%        \If{$M$ $\neq$ target automaton} 
%        \State process counterexample
%        \EndIf
%        \Until{$M$ $=$ target automaton}
%        \EndProcedure
%    \end{algorithmic}
%%\end{algorithm}
%\end{minipage}
%\begin{minipage}{0.5\textwidth}
\centering
\tikzstyle{block} = [rectangle, draw, text width=6em, text centered, rounded corners, minimum height =4em]
\begin{tikzpicture}[->,>=stealth',auto,node distance = 3.5cm]
    \node [draw=none] (0) {};
    \node [block] (1) [right of=0] {is table cohesive?};
    \node [block] (2) [right of=1] {build evidence automaton};
    \node [block] (3) [right of=2,xshift=1.5cm] {build \SFA; is equivalent?};
    \node [block] (4) [right of=3] {done};
    \path 
    (0) edge node[align=center] {initialize\\table} (1)
    (1) edge node {yes} (2)
    (1) edge [loop above] node[align=center] {no:\\membership queries} (1)
    (2) edge node[align=center] {partitioning\\function} (3)
    (3) edge node {yes} (4)
    (3) edge [bend left] node[align=center] {no:\\add counterexample to table} (1)
    ;
\end{tikzpicture}
%\end{minipage}
%\caption{Pseudocode of the learning algorithm \alg (left) and its overview (right).\label{alg:learn}}
\caption{Overview of the learning algorithm \alg.\label{alg:learn}}
\end{figure}

Figure~\ref{alg:learn} shows an overview of the learning algorithm:
after the table is initialized,
the operations \emph{make-consistent},
\emph{evidence-close}, and \emph{close}
are applied until the table is cohesive.\footnote{
It is an invariant of the initialization of the table
and of the operations applied to it
that the observation table is always reduced.}
(\emph{Fill} is applied throughout whenever
a change is made to the table.)
An \SFA $M$ is then conjectured from the table,
and an equivalence query is performed:
if $M$ is equivalent to the target automaton,
then the algorithm terminates.
Otherwise, a counterexample is produced and processed,
and the procedure repeats.

\alg can be thought of as a lazily-evaluated version
of L$^*$ with the additional generalization step,
and therefore it maintains the property of L$^*$
that the learned automaton has a minimal
number of states.

\begin{theorem}[\alg minimality]\label{thm:min}
    When \alg terminates it returns a minimal \SFA.
\end{theorem}

\begin{figure}
    \begin{tabular}{c}
        \begingroup
\scriptsize
\begin{tabular}{r | c}
    $\mathbf{T_1}$ & $\epsilon$ \\ \hline
    $\epsilon$ & + \\ \hline
    0 & +
\end{tabular}
\endgroup

        \begin{tabular}{l}
            \begin{tikzpicture}
    \node[state,accepting] (1) {$\epsilon$};
    \draw[<-] (1) -- node[above] {$\mathbf{M_1^e}$} ++(-.8cm,0);
    \path
    (1) edge [loop above] node {0} (1)
    ;
\end{tikzpicture}
 \\[.75cm]
            \begin{tikzpicture}
    \node[state,accepting] (1) {$\epsilon$};
    \draw[<-] (1) -- node[above] {$\mathbf{M_1}$} ++(-.8cm,0);
    \path
    (1) edge [loop above] node {$[0,\infty)$} (1)
    ;
\end{tikzpicture}

        \end{tabular}
        $\xRightarrow{51~-}$
        \begingroup
\scriptsize
\begin{tabular}{r | c}
    $\mathbf{T_2}$ & $\epsilon$ \\ \hline
    $\epsilon$ & + \\ \hline
    0 & + \\
    51 & -
\end{tabular}
\endgroup
 
        $\xRightarrow{\text{close}}$
        \begingroup
\scriptsize
\begin{tabular}{r | c}
    $\mathbf{T_3}$ & $\epsilon$ \\ \hline
    $\epsilon$ & + \\
    51 & - \\ \hline
    0 & + \\
    51,0 & -
\end{tabular}
\endgroup

        \begin{tabular}{l}
            \input{f/m3e} \\[.75cm]
            \input{f/m3s}
        \end{tabular} 
        $\xRightarrow{101~+}$        
        \\[1.6cm]
        %
        \begingroup
\scriptsize
\begin{tabular}{r | c}
    $\mathbf{T_4}$ & $\epsilon$ \\ \hline
    $\epsilon$ & + \\
    51 & - \\ \hline
    0 & + \\
    51,0 & - \\
    101 & +
\end{tabular}
\endgroup

        \begin{tabular}{r}
            \begin{tikzpicture}
    \node[state,accepting] (1) {$\epsilon$};
    \draw[<-] (1) -- node[above] {$\mathbf{M_4^e}$} ++(-.8cm,0);
    \node[state] (2) [right of=1] {51};
    \path
    (1) edge [loop above] node {0,101} (1)
    (1) edge node {51} (2)
    (2) edge [loop above] node {0} (2)
    ;
\end{tikzpicture}
 \\[.75cm]
            \begin{tikzpicture}
    \node[state,accepting] (1) {$\epsilon$};
    \draw[<-] (1) -- node[above] {$\mathbf{M_4}$} ++(-.8cm,0);
    \node[state] (2) [right of=1] {51};
    \path
    (1) edge [loop above] node {$[0,51) \cup [101,\infty)$} (1)
    (1) edge node {$[51,101)$} (2)
    (2) edge [loop above] node {$[0,\infty)$} (2)
    ;
\end{tikzpicture}

        \end{tabular}
        $\xRightarrow{51,0,0~+}$
        \begingroup
\scriptsize
\begin{tabular}{r | c}
    $\mathbf{T_5}$ & $\epsilon$ \\ \hline
    $\epsilon$ & + \\
    51 & - \\ \hline
    0 & + \\
    51,0 & - \\
    101 & + \\
    51,0,0 & +
\end{tabular}
\endgroup

        $\xRightarrow{
        \begin{scriptsize}                                     
        \begin{array}{c}
         \text{make}\\
         \text{consistent}
		\end{array}
		\end{scriptsize}    
        }$
        \begingroup
\scriptsize
\begin{tabular}{r | c c}
    $\mathbf{T_6}$ & $\epsilon$ & 0\\ \hline
    $\epsilon$ & + & + \\
    51 & - & - \\ \hline
    0 & + & + \\
    51,0 & - & + \\
    101 & + & + \\
    51,0,0 & + & +
\end{tabular}
\endgroup
 
        $\xRightarrow{\text{close}}$             
        \\[1.6cm]
        %
        \hspace{-7mm}
        \begingroup
\scriptsize
\begin{tabular}{r | c c}
    $\mathbf{T_7}$ & $\epsilon$ & 0\\ \hline
    $\epsilon$ & + & + \\
    51 & - & - \\
    51,0 & - & + \\ \hline
    0 & + & + \\
    101 & + & + \\
    51,0,0 & + & +
\end{tabular}
\endgroup
\hspace{-1mm}\begin{tikzpicture}
    \node[state,accepting] (1) {$\epsilon$};
    \draw[<-] (1) -- node[above] {$\mathbf{M_7^e}$} ++(-.8cm,0);
    \node[state] (2) [right of=1] {51};
    \node[state] (3) [below of=1] {51,0};
    \path
    (1) edge [loop above] node {0,101} (1)
    (1) edge node {51} (2)
    (2) edge [bend right=20] node {0} (3)
    (3) edge node {0} (1)
    ;
\end{tikzpicture}
\hspace{-2mm}\input{f/m7s}
        $\xRightarrow{
        \begin{scriptsize}                                     
        \begin{tiny}
        \begin{array}{c}
         51,21,0\\
         -
		\end{array}
        \end{tiny}		
		\end{scriptsize}    
        }$
        \begingroup
\scriptsize
\begin{tabular}{r | c c}
    $\mathbf{T_8}$ & $\epsilon$ & 0\\ \hline
    $\epsilon$ & + & + \\
    51 & - & - \\
    51,0 & - & + \\ \hline
    0 & + & + \\
    101 & + & + \\
    51,0,0 & + & + \\
    51,21 & - & - \\
    51,21,0 & - & -
\end{tabular}
\endgroup

        $\xRightarrow{
        \begin{scriptsize}                                     
        \begin{array}{c}
         \text{mk}\\
         \text{cons.}
		\end{array}
		\end{scriptsize}    
        }$
        \begingroup
\scriptsize
\begin{tabular}{r | c c c}
    $\mathbf{T_9}$ & $\epsilon$ & 0 & 0,0\\ \hline
    $\epsilon$ & + & + & + \\
    51 & - & - & + \\
    51,0 & - & + & + \\ \hline
    0 & + & + & + \\
    101 & + & + & + \\
    51,0,0 & + & + & + \\
    51,21 & - & - & - \\
    51,21,0 & - & - & - \\
    0,0 & + & + & + \\
    51,0,0,0 & + & + & +
\end{tabular}
\endgroup
\\[1.6cm]
        %         
        $\xRightarrow{\text{close}}$
        \begingroup
\scriptsize
\begin{tabular}{r | c c c}
    $\mathbf{T_{10}}$ & $\epsilon$ & 0 & 0,0\\ \hline
    $\epsilon$ & + & + & + \\
    51 & - & - & + \\
    51,0 & - & + & + \\
    51,21 & - & - & - \\ \hline
    0 & + & + & + \\
    101 & + & + & + \\
    51,0,0 & + & + & + \\
    51,21,0 & - & - & - \\
    0,0 & + & + & + \\
    51,0,0,0 & + & + & + \\
    51,21,0,0 & - & - & -
\end{tabular}
\endgroup
 \input{f/m10e}\hspace{-1mm}\input{f/m10s} \\[2cm]
        %
        $\xRightarrow{51,0,21~-}$
        \input{f/t11} \input{f/m11e} \begin{tikzpicture}
    \node[state,accepting] (1) {$\epsilon$};
    \draw[<-] (1) -- node[above] {$\mathbf{M_{11}}$} ++(-.8cm,0);
    \node[state] (2) [right of=1] {51};
    \node[state] (3) [below of=1] {51,0};
    \node[state] (4) [below of=2] {51,21};
    \path
    (1) edge [loop above] node {$[0,51) \cup [101,\infty)$} (1)
    (1) edge node {$[51,101)$} (2)
    (2) edge [bend right=20] node {$[0,21)$} (3)
    (2) edge node {$[21,\infty)$} (4)
    (3) edge node {$[0,21)$} (1)
    (3) edge node {$[21,\infty)$} (4)
    (4) edge [loop right] node {$[0,\infty)$} (4)
    ;
\end{tikzpicture}

    \end{tabular}
    \caption{An example run of the \alg\ algorithm.}
    \label{fig:example}
\end{figure}


\subsection{Worked Example}\label{subsec:ex}
Suppose we invoke \alg\ to learn the automaton
over non-negative integers that accepts
all words \emph{except} those that
contain a number between 51 and 100 that
is not immediately followed by two numbers
between 0 and 20.

The Boolean algebra we use is the union
of left-closed right-open intervals.
We fix a partitioning function $P$ that assumes
that if $a \in \ell$, $b \in \ell'$,
and there are no $c$ in the input sets
such that $a < c < b$, then the whole
interval $[a,b)$ behaves equivalently to $a$.
For example,
$P(\{0\},\{10\}) = [0,10), [10,\infty)$
and
$P(\{0,20\},\{10\}) = [0,10)\cup[20,\infty), [10,\infty)$.

The trace of the algorithm is illustrated
in Figure~\ref{fig:example}.
\alg\ begins by initializing an observation table
so that $S = \{\epsilon\}$, $R = \{0\}$
(the 0 is an arbitrary character from $\Sigma$
and is used purely so that the table contains
$\epsilon \cdot a$ for some $a$),
and $E = \{\epsilon\}$.
The appropriate membership queries are
made to the oracle, resulting in the table $\mathbf{T_1}$.
$\mathbf{T_1}$ is cohesive, so it is used to construct the evidence automaton
$\mathbf{M_1^e}$, and by calling the partitioning function
$P$ on the outgoing transitions of each state in $\mathbf{M_1^e}$---in
this case just $P(\{0\}) = [0,\infty)$---the
\SFA\ $\mathbf{M_1}$ is conjectured.
The oracle is given $\mathbf{M_1}$ as an equivalence query,
and it returns the single-character word 51
as a counterexample.
51 is added to $R$ in the observation table,
as would all of its prefixes if it were a word of length
greater than one,
and a membership query is asked for $51\cdot\epsilon$,
resulting in table $\mathbf{T_2}$.

$\mathbf{T_2}$ is not closed, since $\textit{row}(51) = -$
and there is no $s \in S$ with $\textit{row}(s) = -$.
Accordingly, 51 represents a path to a new state,
so it is moved from $S$ to $R$, and a continuation
$51,0$ is added to $R$.  This produces
table $\mathbf{T_3}$, which is now cohesive
and thus admits the construction
of the evidence automaton $\mathbf{M_3^e}$
and ultimately the \SFA\ $\mathbf{M_3}$
through the use of the partitioning function:
for example, for the outgoing transitions of the initial state,
$P(\{0\},\{51\}) = [0,51), [51,\infty)$.
An equivalence query sent to the oracle
returns the counterexample of 101.

Adding 101 to $R$ results in the
cohesive table $\mathbf{T_4}$ and the \SFA\ $\mathbf{M_4}$,
and the oracle provides the counterexample $51,0,0$.
$51,0,0$ is added to $R$ (all of its prefixes
are already present in $S \cup R$),
resulting in the table $\mathbf{T_5}$ which is not
consistent:
observe that $\textit{row}(51) = - = \textit{row}(51,0)$,
but $\textit{row}(51\cdot0) = - \neq + = \textit{row}(51,0\cdot0)$.
This means that $51$ and $51,0$ actually lead
to different states, which will be addressed in two stages.
First, following the rule \emph{make-consistent}, since
$f(51\cdot0\cdot\epsilon) \neq f(51,0\cdot0\cdot\epsilon)$,
we add $0\cdot\epsilon$ to $E$ to distinguish
the states led to by $51$ and $51,0$,
which produces table $\mathbf{T_6}$.
Applying \emph{close} to $\mathbf{T_6}$ results in $\mathbf{T_7}$,
which is then cohesive
(we added an element to $E$, which would
normally require applying \emph{evidence-close},
but it happens to be that $\mathbf{T_7}$ is already
evidence-closed) and produces
an \SFA\ $\mathbf{M_7}$.
The counterexample $51,21,0$
requires adding it as well as the prefix $51,21$
to $R$, producing table $\mathbf{T_8}$.

$\mathbf{T_8}$ is also inconsistent,
since $\textit{row}(51) = -,- = \textit{row}(51,21)$
but $\textit{row}(51\cdot0) = -,+ \neq -,- = \textit{row}(51,21\cdot0)$.
Since $f(51\cdot0\cdot0) \neq f(51,21\cdot0\cdot0)$,
we add $0\cdot0$ to $E$
to distinguish $51$ from $51,21$,
and evidence-close the table to get $\mathbf{T_9}$.
Closing and evidence-closing this table results in $\mathbf{T_{10}}$,
the conjecture $\mathbf{M_{10}}$, the counterexample
$51,0,21$, the table $\mathbf{T_{11}}$, and finally
the automaton $\mathbf{M_{11}}$ which passes
the equivalence query.
