\section{Learnability and its Properties}
\label{sec:learnability}

Whether an \SFA can be learned and, if so, the complexity of learning that \SFA,
depends on a more fundamental property concerning the \emph{learnability}
of the underlying Boolean algebra.
In \alg,
this notion of learnability determines the
complexity of the algorithm.
We first provide a definition for an algebra's learnability
with respect to the inputs given to a partitioning function
and then connect these inputs to the queries given to the oracle
during the learning algorithm.

\subsection{Learnability of a Boolean Algebra}

Fix a partitioning function $P$ over a Boolean algebra $\mathcal{A}$.
Let $C$ denote a concept class for which each concept $c \in C$ is a finite partition
of $\mathfrak{D}_\mathcal{A}$ using predicates in $\Psi_\mathcal{A}$,
and let $G$ denote the set of \emph{generators} which, informally,
provide a sequence of counterexamples---elements
in $\mathfrak{D}_\mathcal{A}$---to update the sets given to $P$.
We analyze how many times a generator $g$ must make an update
before $P$ learns a desired partition.
A generator $g \in G$ can be thought of as a function
that takes as input
a tuple $(L, c_\text{guess}, c_\text{target})$---where
$L$ is the list of subsets of $\mathfrak{D}_\mathcal{A}$
given as input to $P$,
$c_\text{guess} \in C$ is a partition of $\mathfrak{D}_\mathcal{A}$
consistent with $L$, and
$c_\text{target} \in C$ is the target partition---and
outputs a new list $L'$ of $\mathfrak{D}_\mathcal{A}$-subsets.
We say $g$ provides sets to $P$ to refer to
the iterative process in which
$L_0 = \lbrack \emptyset \rbrack$ and
$L_{i+1} = g(L_i, P(L_i), c_\text{target})$.
Intuitively, a generator iteratively updates a list of sets
to be given to a partitioning function so that
the output of that function approaches the target partition.

Additionally, the generators are subject to the following restrictions
that ensure a sense of monotonicity:
%
\rone the output $L'$ is \emph{greater than}
the input $L$ in the sense that 
$\forall a \in \mathfrak{D}_\mathcal{A}
\lbrack (\exists \ell \in L \ldotp a \in \ell) \rightarrow
(\exists \ell' \in L' \ldotp a \in \ell')\rbrack$
(a character present in the input
will always be present in future iterations);
\rtwo if $a_1 \in \ell_i \in L$ and $a_2 \in \ell_j \in L$
and $i \neq j$, then
it cannot be that there is some $\ell' \in L'$ and
both $a_1 \in \ell'$ and $a_2 \in \ell'$
(if the generator says two elements belong to different
sets in a partition, that must be true for all future iterations);
and \rthree either
the number of sets in $L'$ is larger than
the number of sets in $L$, or
at least one $a \in \mathfrak{D}_\mathcal{A}$
that was not present in any $\ell \in L$ is present in some $\ell' \in L'$
%
Also, the inputs to the generator are subject to a notion of consistency:
if $a_1 \in \ell_i \in L$ and $a_2 \in \ell_j \in L$
such that $i \neq j$, then there is no
$\varphi \in c_\text{target}$ such that
$\{a_1, a_2\} \subseteq \den{\varphi}$.

This definition of a generator exactly captures
the high-level process of updating the observation
table in our algorithm via queries to the oracle
and \emph{projecting} those changes onto
the individual lists of sets that are given
to the partitioning function for the creation
of the conjectured \SFA.
For example, in Figure~\ref{fig:example},
the evidence for the outgoing transitions
of the $\epsilon$-state is provided by a generator such that
$L_1 = \lbrack\{0\}\rbrack$,
$L_2 = \lbrack\{0\}, \{51\}\rbrack$,
and $L_3 = \lbrack\{0,101\},\{51\}\rbrack$.
Below we will formalize
a notion of learnability with respect
to these generators, and it will thus
bear a close correspondence to the
complexity of learning an \SFA.

\begin{definition}[$s_g$-learnability]
    Given a Boolean algebra $\mathcal{A}$,
    a partitioning function $P$, and
    a generator $g \in G$,
    we say the pair $(\mathcal{A}, P)$ is \emph{$s_g$-learnable}
    if there exists an implicit function
    $s_g : C \rightarrow \mathbb{N}$
    such that $P$ needs as input a list of sets,
    provided by $g$,
    with total size at most $s_g(c)$
    to discover a target partition $c \in C$.
    Furthermore, we say $\mathcal{A}$ itself
    is $s_g$-learnable if there exists
    a partitioning function $P$
    such that $(\mathcal{A}, P)$ is $s_g$-learnable.
\end{definition}

We also classify $\mathcal{A}$
as belonging to a \emph{learning class}
that depends on these $s_g$ functions---but
first we need an auxiliary notion of the \emph{size}
of a partition.

\begin{definition}[DNF-Size of a partition]
    Let $C$ be the set of partitions of $\mathcal{A}$.
    Each $c \in C$ is a list $\varphi_1, \ldots, \varphi_n$:
    we can expand each $\varphi_i$ to a minimal
    disjunctive-normal-form formula $\bigvee_j \psi_{i,j}$
    such that
    $c' = \psi_{1,1}, \ldots, \psi_{1,m_1}, \ldots, \psi_{n,1}, \ldots, \psi_{n,m_n}$
    is a partition of $\mathcal{A}$ that is
    at least as fine as $c$.
    We say the \emph{DNF-size} of $c$
    is the length of the list of such a minimal $c'$.
\end{definition}

\begin{example}
    The partition $\{x < 5 \lor x > 10, 5 \leq x \land x \leq 10\}$
    has DNF-size 3.
\end{example}

\begin{definition}[Learning Class]
    For a fixed Boolean algebra $\mathcal{A}$
    if there exists a $g \in G$ such that
    $\mathcal{A}$ is $s_g$-learnable, then
    \begin{itemize}
        \item if $s_g$ is a constant function,
            i.e. $\exists k \forall c \ldotp s_g(c) = k$,
            we say $\mathcal{A} \in \mathcal{C}_\textit{const}^\exists$
        \item if $s_g$ is a function only of the DNF-size of $c$,
            we say $\mathcal{A} \in \mathcal{C}_\textit{size}^\exists$
        \item if $s_g$ is otherwise unconstrained,
            we say $\mathcal{A} \in \mathcal{C}_\textit{finite}^\exists$
    \end{itemize}

    Additionally, for some fixed partitioning function $P$,
    if for all $g \in G$,
    $(\mathcal{A}, P)$ is $s_g$-learnable, then
    \begin{itemize}
        \item if each $s_g$ is a constant function,
            we say $\mathcal{A} \in \mathcal{C}_\textit{const}^\forall$
        \item if each $s_g$ is a function only of the DNF-size of $c$,
            we say $\mathcal{A} \in \mathcal{C}_\textit{size}^\forall$
        \item if each $s_g$ is otherwise unconstrained,
            we say $\mathcal{A} \in \mathcal{C}_\textit{finite}^\forall$
    \end{itemize}
\end{definition}
%

\begin{wrapfigure}{r}{0.35\textwidth}
\vspace{-4mm}
\centering
\begin{tabular}{l l l l l}
    $\mathcal{C}_\textit{const}^\forall$ & $\subseteq$~ & $\mathcal{C}_\textit{size}^\forall$ & $\subseteq$~ & $\mathcal{C}_\textit{finite}^\forall$ \\
    ~\rotatebox[origin=c]{-90}{$\subseteq$} & & ~\rotatebox[origin=c]{-90}{$\subseteq$} & & ~\rotatebox[origin=c]{-90}{$\subseteq$} \\
    $\mathcal{C}_\textit{const}^\exists$ & $\subseteq$~ & $\mathcal{C}_\textit{size}^\exists$ & $\subseteq$~ & $\mathcal{C}_\textit{finite}^\exists$ \\
\end{tabular}
\caption{Learning classes. \label{fig:relation}}
\vspace{-5mm}
\end{wrapfigure}
Observe that learning classes are partially-ordered by the subset relation
shown in Figure~\ref{fig:relation}.
This categorization is convenient for reasoning about
different instantiations of domains and oracles.
For example:
%
\rone When $\mathcal{A} \in \mathcal{C}_\textit{const}^\forall$, learning a
partition over $\mathfrak{D}_\mathcal{A}$ is equivalent
to the machine-learning notion of a mistake-bound~\cite{littlestone88}.
%
\rtwo The equality algebra for any finite alphabet is in $\mathcal{C}_\textit{const}^\forall$.
%
\rthree The interval algebra over the integers or rationals
is in $\mathcal{C}_\textit{size}^\exists$;
if the oracle provides lexicographically minimal counterexamples,
the number of times the partitions must be updated
through the generator is determined by the number of
connected regions in the partition,
as illustrated in~\cite{mens15}
and as applicable for Figure~\ref{fig:example}.
%
The integer case is also in $\mathcal{C}_\textit{finite}^\forall$,
since after arbitrary counterexamples are found beyond the least
and greatest finite bounds in the partition, $m$ and $M$ respectively,
at most $M - m$ more counterexamples are required.
%
\rfour Using enumeration, linear rational arithmetic is in $\mathcal{C}_\textit{finite}^\forall$.
%


Since for each state in an \SFA, the set of outgoing transitions
forms a partition of the alphabet, i.e. a concept in $C$,
the number of counterexamples
needed to learn the entire automaton is related to
the sum of $s_g(c)$ for each state's outgoing transitions.
Hence, the complexity of learning depends on
\rone the choice of the partitioning function and, potentially,
\rtwo the quality of counterexamples provided by the oracle.

\begin{theorem}[SFA Learnability]\label{thm:learn}
    If $M$ is an \SFA over a learnable Boolean algebra $\mathcal{A}$ with $n$ states,
    then the number of equivalence queries
    needed to learn $M$ is bounded above by $n^2 \sum_{q_i \in Q} s_{g_i}(c_i)$,
    where $s_{g_i}$ is the projection of the oracle
    onto learning the partition $c_i$
    for the outgoing transitions of state $q_i$.
\end{theorem}

The notion of an algebra's learning class can have powerful ramifications
in conjuction with the result of Theorem~\ref{thm:learn}.
For example, if an \SFA uses a Boolean algebra contained
in $\mathcal{C}_\textit{finite}^\forall$, then the use
of the appropriate partitioning function guarantees termination
of the learning algorithm, independent of the quality of counterexamples
produced from equivalence queries.
%The automaton $\mathbf{M_{11}}$ from Figure~\ref{fig:example}
%learned with the partitioning function from Section~\ref{subsec:ex}
%and an oracle that provides minimal counterexamples instantitates
%$\mathcal{C}_\textit{size}^\exists$, and for this automaton
%$n = 4$ and $\sum_i s_{g_i}(c_i) = 8$; consequently,
%the number of equivalence queries utilized by the algorithm
%is pessimistically bounded from above by 128.
%%%%the bound is potentially taken from the max, not the sum
Investigating which of the subset relations in Figure~\ref{fig:relation}
are strict subsets, as well as what (if any) algebras fall outside
of $\mathcal{C}_\textit{finite}^\exists$ are interesting
future problems.

\subsection{Composing Learnable Algebras}\label{sec:comp}

The definition of learnability described prior has
the remarkable property that it is preserved by some 
constructions that combine Boolean algebras,
such as the \emph{disjoint union} and the \emph{cartesian product}.
In these cases, a partitioning function for the 
resultant algebra can be constructed by using partitioning
functions for the original algebras as black boxes;
This allows us to phrase
the learnability of the constructed algebra
in terms of the learnability of the individual algebras.

\begin{definition}[Disjoint Union Algebra]
    Let $\mathcal{A}_1, \mathcal{A}_2$ be boolean algebras.
    Their \emph{disjoint union algebra}
    $\mathcal{A}_\uplus = (\mathfrak{D}, \Psi, \den{\_}, \bot, \top, \vee, \wedge, \neg)$,
    which we denote $\mathcal{A}_\uplus = \mathcal{A}_1 \uplus \mathcal{A}_2$,
    is constructed as follows:\footnote{In our definition, we use
    $\mathfrak{D}_{\mathcal{A}_1} \uplus \mathfrak{D}_{\mathcal{A}_2}$
    to denote the disjoint union of sets; rigorously, 
    when the sets are not already disjoint,
    this is constructed by taking
    $(\mathfrak{D}_{\mathcal{A}_1} \times \{1\}) \cup
    (\mathfrak{D}_{\mathcal{A}_2} \times \{2\})$ and lifting
    all the remaining constructs appropriately.}
    $$
    \begin{array}{c}
        \mathfrak{D} = \mathfrak{D}_{\mathcal{A}_1} \uplus \mathfrak{D}_{\mathcal{A}_2} \qquad
        \Psi = \Psi_{\mathcal{A}_1} \times \Psi_{\mathcal{A}_2} \qquad
        \den{(\varphi_{\mathcal{A}_1}, \varphi_{\mathcal{A}_2})} = 
            \den{\varphi_{\mathcal{A}_1}}_{\mathcal{A}_1} \uplus
            \den{\varphi_{\mathcal{A}_2}}_{\mathcal{A}_2} \\
        \bot = (\bot_{\mathcal{A}_1}, \bot_{\mathcal{A}_2}) \qquad
        \top = (\top_{\mathcal{A}_1}, \top_{\mathcal{A}_2}) \qquad
        \neg (\varphi_{\mathcal{A}_1}, \varphi_{\mathcal{A}_2}) =
            (\neg_{\mathcal{A}_1} \varphi_{\mathcal{A}_1},
            \neg_{\mathcal{A}_2} \varphi_{\mathcal{A}_2})\\
        (\varphi_{\mathcal{A}_1}, \varphi_{\mathcal{A}_2}) \vee
            (\varphi_{\mathcal{A}_1}', \varphi_{\mathcal{A}_2}') = 
            ((\varphi_{\mathcal{A}_1} \vee_{\mathcal{A}_1} \varphi_{\mathcal{A}_1}'),
            (\varphi_{\mathcal{A}_2} \vee_{\mathcal{A}_2} \varphi_{\mathcal{A}_2}')) \\
        (\varphi_{\mathcal{A}_1}, \varphi_{\mathcal{A}_2}) \wedge
            (\varphi_{\mathcal{A}_1}', \varphi_{\mathcal{A}_2}') = 
            ((\varphi_{\mathcal{A}_1} \wedge_{\mathcal{A}_1} \varphi_{\mathcal{A}_1}'),
            (\varphi_{\mathcal{A}_2} \wedge_{\mathcal{A}_2} \varphi_{\mathcal{A}_2}')) \\        
    \end{array}
    $$
\end{definition}

If $\mathcal{A}_1$ has partitioning function $P_1$
and $\mathcal{A}_2$ has partitioning function $P_2$,
then we can construct a partitioning function $P_\uplus$
for $\mathcal{A}_\uplus = \mathcal{A}_1 \uplus \mathcal{A}_2$:
$P_\uplus$ takes as input a list $L_\uplus$ of sets
where each set $\ell_{\uplus_i} \subset \mathfrak{D}_{\mathcal{A}_1} \uplus \mathfrak{D}_{\mathcal{A}_2}$.
We decompose $L_\uplus$ into $L_{\mathfrak{D}_1}$ and $L_{\mathfrak{D}_2}$, two lists of sets
of $\ell_{1_i} \subset \mathfrak{D}_{\mathcal{A}_1}$
and $\ell_{2_i} \subset \mathfrak{D}_{\mathcal{A}_2}$, respectively:
for each $a \in \ell_{\uplus_i}$, 
if $a \in \mathfrak{D}_{\mathcal{A}_1}$, then we add $a$ to $\ell_{1_i}$, and otherwise
if $a \in \mathfrak{D}_{\mathcal{A}_2}$, then we add $a$ to $\ell_{2_i}$.
We obtain $L_{\Psi_1} = P_1(L_{\mathfrak{D}_1})$
and $L_{\Psi_2} = P_2(L_{\mathfrak{D}_2})$.
We construct $L_{\Psi_\uplus}$ by taking
$\varphi_{\uplus_i} = (\varphi_{1_i}, \varphi_{2_i})$ for all $i$,
return $L_{\Psi_\uplus}$, and terminate.

The disjoint union is useful since, for example,
we can represent arbitrary intervals over the integers
as the disjoint union of \rone intervals over non-negative integers
and \rtwo intervals over negative integers.
In other words, a partitioning function suited
for a single notion of $\infty$ can be extended
to capture two.

\begin{theorem}[Disjoint Union Algebra Learnability]\label{thm:du}
    Given Boolean algebras $\mathcal{A}_1, \mathcal{A}_2$
    with partitioning functions $P_1, P_2$,
    $(\mathcal{A}_1, P_1)$ is $s_{g_1}$-learnable
    and $(\mathcal{A}_2, P_2)$ is $s_{g_2}$-learnable
    if and only if there exists $g_\uplus$ such that their disjoint union algebra
    $(\mathcal{A}_\uplus, P_\uplus)$ is $s_{g_\uplus}$-learnable,
    where $s_{g_\uplus}(c) = s_{g_1}(c_1) + s_{g_2}(c_2)$
    and $c_1$ and $c_2$ are the restrictions of $c$
    to $\mathfrak{D}_{\mathcal{A}_1}$ and $\mathfrak{D}_{\mathcal{A}_2}$,
    respectively.
\end{theorem}

\begin{corollary}\label{thm:ducor}
    If $\mathcal{A}_1$
    and $\mathcal{A}_2$ are in learning class $\mathcal{C}$,
    then their disjoint union $\mathcal{A}_\uplus$ is also in learning class
    $\mathcal{C}$.
\end{corollary}

We present a similar construction for the \emph{product}
of two Boolean algebras.

\begin{definition}[Product Algebra]
    Let $\mathcal{A}_1, \mathcal{A}_2$ be boolean algebras.
    Their \emph{product algebra}
    $\mathcal{A}_\times = (\mathfrak{D}, \Psi, \den{\_}, \bot, \top, \vee, \wedge, \neg)$,
    which we denote $\mathcal{A}_\times = \mathcal{A}_1 \times \mathcal{A}_2$,
    is constructed as follows:
    $$\begin{array}{c}
        \mathfrak{D} = \mathfrak{D}_{\mathcal{A}_1} \times \mathfrak{D}_{\mathcal{A}_2} \quad
        \Psi = 2^{\Psi_{\mathcal{A}_1} \times \Psi_{\mathcal{A}_2}} \quad
        \den{\{(\varphi_{\mathcal{A}_1i}, \varphi_{\mathcal{A}_2i})\}_i} = 
            \textstyle\bigcup_i ~\den{\varphi_{\mathcal{A}_1i}}_{\mathcal{A}_1}
            \times \den{\varphi_{\mathcal{A}_2i}}_{\mathcal{A}_2} \\
        \bot = \{(\bot_{\mathcal{A}_1}, \bot_{\mathcal{A}_2})\} \qquad
        \top = \{(\top_{\mathcal{A}_1}, \bot_{\mathcal{A}_2})\} \\
        \neg\{(\varphi_{\mathcal{A}_1i}, \varphi_{\mathcal{A}_2i})\}_i =
            \textstyle\bigwedge_i\{(\neg_{\mathcal{A}_1} \varphi_{\mathcal{A}_1i}, \top_{\mathcal{A}_2}),
            (\top_{\mathcal{A}_1}, \neg_{\mathcal{A}_2} \varphi_{\mathcal{A}_2i})\}\\        
        \{(\varphi_{\mathcal{A}_1i}, \varphi_{\mathcal{A}_2i})\}_i \vee
            \{(\varphi_{\mathcal{A}_1j}', \varphi_{\mathcal{A}_2j}')\}_j = 
            \{(\varphi_{\mathcal{A}_1i}, \varphi_{\mathcal{A}_2i})\}_i \cup
            \{(\varphi_{\mathcal{A}_1j}', \varphi_{\mathcal{A}_2j}')\}_j \\ 
        \{(\varphi_{\mathcal{A}_1i}, \varphi_{\mathcal{A}_2i})\}_i \wedge
            \{(\varphi_{\mathcal{A}_1j}', \varphi_{\mathcal{A}_2j}')\}_j = 
            \{(\varphi_{\mathcal{A}_1i} \wedge_{\mathcal{A}_1} \varphi_{\mathcal{A}_1j}',
            \varphi_{\mathcal{A}_2i} \wedge_{\mathcal{A}_2} \varphi_{\mathcal{A}_2j}')
            \mid \forall i,j\} \\
    \end{array}$$
\end{definition}

If $\mathcal{A}_1$ has partitioning function $P_1$
and $\mathcal{A}_2$ has partitioning function $P_2$,
then we can construct a partitioning function $P_\times$
for $\mathcal{A}_\times = \mathcal{A}_1 \times \mathcal{A}_2$:
$P_\times$ takes as input a list $L_\times$ of sets
where each set $\ell_{\times_i} \subset \mathfrak{D}_{\mathcal{A}_1} \times \mathfrak{D}_{\mathcal{A}_2}$.
%
We first take the set $D_1 = \{d_1 \mid (d_1, d_2) \in \ell_{\times_i}
\text{ for some } \ell_{\times_i} \in L_\times \}$,
turn it into a list $D_1' = \{d_{1,1}\},\ldots,\{d_{1,n}\}$,
and compute a partition $L_1 = P_1(D_1')$.
Then for each $d_i \in D_1$, we construct a list
$D_{2,d_i}$ where the $j$-th element is the set
$\{d_2 \mid (d_i, d_2) \in \ell_{\times_j}\}$
and compute a partition $L_{2,d_i} = P_2(D_{2,d_i})$.
%
Finally, we initialize the list of predicates to be returned
$L_{\Psi_\times} = \varphi_{\times_1}, \ldots, \varphi_{\times_k}$
so that initially each $\varphi_{\times_i} = \bot$.
Then for all $i$ and each $(d_1, d_2) \in \ell_{\times_i}$,
let $\varphi_{d_1}$ be the predicate in $L_1$ corresponding to
$\{d_1\}$ in $D_1'$
and let $\varphi_{d_2}$ be the predicate in $L_{2,d_1}$
corresponding to the set of $D_{2,d_1}$ that contains $d_2$;
update $\varphi_{\times_i} \gets \varphi_{\times_i} \vee (\varphi_{d_1}, \varphi_{d_2})$.
%
Return $L_{\Psi_\times}$ and terminate.

\begin{example}
    Suppose we want to find a partition over $(x,y) \in \mathbb{Z} \times \mathbb{Z}$
    where each component uses the interval algebra,
    and suppose the input sets are
    $L_\times = [\{(0, 0), (1, 0), (1,2)\}, \{(0,2)\}]$.
    Then $D_1' = [\{0\}, \{1\}]$
    and perhaps $L_1 = P_1(D_1') = [x \leq 0, x > 0]$.
    Then we have
    $D_{2,0} = [\{0\}, \{2\}]$ and
    $D_{2,1} = [\{0,2\}, \emptyset]$.
    Perhaps
    $L_{2,0} = P_2(D_{2,0}) = [y \leq 1, y > 1]$ and
    $L_{2,1} = P_2(D_{2,1}) = [\top, \bot]$.
    Then (without simplification)
    $L_{\Psi_\times} = [(x \leq 0, y \leq 1) \vee (x > 0, \top) \vee (x > 0, \top), (x \leq 0, y > 1)]$
\end{example}

\begin{theorem}[Product Algebra Learnability]\label{thm:prod}
    Given Boolean algebras $\mathcal{A}_1, \mathcal{A}_2$
    with partitioning functions $P_1, P_2$
    and their product algebra $\mathcal{A}_\times$
    with the composite partitioning function $P_\times$,
    let $c \in C_\times$ be the target partition over the product algebra,
    let $c_1 \in C_1$ be the minterms of the $\mathcal{A}_1$-components of $c$, and
    let $c_2 \in C_2$ be the minterms of the $\mathcal{A}_2$-components of $c$.
    \rone If $(\mathcal{A}_1, P_1)$ is $s_{g_1}$-learnable
    and $(\mathcal{A}_2, P_2)$ is $s_{g_2}$-learnable,
    then there exists $g_\times$ such that
    $(\mathcal{A}_\times, P_\times)$ is $s_{g_\times}$-learnable
    where $s_{g_\times}(c) = s_{g_1}(c_1) s_{g_2}(c_2)$.
    \rtwo If $(\mathcal{A}_\times, P_\times)$ is $s_{g_\times}$-learnable,
    then there exist $g_1, g_2$ such that
    $(\mathcal{A}_1, P_1)$ is $s_{g_1}$-learnable
    and $(\mathcal{A}_2, P_2)$ is $s_{g_2}$-learnable
    where $s_{g_\times}(c) = s_{g_1}(c_1) = s_{g_2}(c_2)$.
\end{theorem}

\begin{corollary}\label{thm:prodcor}
    If $\mathcal{A}_1$
    and $\mathcal{A}_2$ are in learning class $\mathcal{C}$,
    then their product $\mathcal{A}_\times$ is also in learning class
    $\mathcal{C}$.
\end{corollary}
Since learnability is closed
under disjoint union and product,
symbolic automata over non-recursive data types can 
be learned using partitioning functions for the component types,
as opposed to necessitating specialized partitioning functions.

