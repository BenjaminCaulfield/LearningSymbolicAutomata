\section{Implementation}
\label{sec:implementation}

We implemented \alg in the open-source Java library
Symbolic Automata.
Our modular implementation only requires the programmer to provide
learnable Boolean algebras as input to the learner;
we have already implemented the equality and interval algebras
as well as the disjoint union and product algebras---which
are implemented as meta-algebras and can be instantiated arbitrarily.

We evaluated our algorithm on the examples  presented by
Maler and Mens~\cite{mens15}, who
proposed two extensions of L$^*$
for learning \SFAs where 1) predicates are
union of intervals in $\mathbb{N}$, or
2) predicates are
union of intervals over $\mathbb{N}\times \mathbb{N}$.
Their algorithms  assume that the oracle always provides lexicographically minimal counterexamples,
so that every counterexample identifies a boundary in a partition.
They evaluate their techniques on two automata: one over 
the alphabet $\mathbb{N}$ (Ex. 4.1~\cite{mens15})  and one over the alphabet $\mathbb{N} \times \mathbb{N}$
(Ex. 5.1~\cite{mens15}).

We implemented a partitioning function equivalent to their characterization
of the interval algebra over $\mathbb{N}$.
While, to learn automata over $\mathbb{N} \times \mathbb{N}$,
\cite{mens15} introduces an ad-hoc separate technique that  requires the oracle
to always give locally minimal counterexamples,
in our setting, the algebra for pairs can be trivially implemented as
the Cartesian product of the interval algebra over $\mathbb{N}$ with itself.

We learn the first automaton using 8 equivalence and 23 membership queries,
while their algorithm only requires 7 and 17, respectively.
The former difference is due to their algorithm adding a different suffix
to $E$ than ours, which happens to discover two new states instead of one
and ultimately saves them an equivalence query.
The latter is due to a more refined handling of counterexamples (more in our related work).
Similarly,  we learn the second 
automaton using 28 equivalence and 43 membership queries,
while their algorithm only requires
%We learned this automaton by using the
%meta-partitioning function for product algebras
%presented in Section~\ref{sec:comp},
%which is able to learn the automaton
18 and 20, respectively.
In this case, the discrepancy is amplified because
the algorithm in~\cite{mens15} uses a specialized
implicit partitioning function that avoids the quadratic
blowup caused by the Cartesian product construction
in Theorem~\ref{thm:prod}.
%suggests that using the meta-algebra should
%require about quadratically more
%equivalence queries.
We implemented an analogous specialized partitioning function
directly on the product algebra and were able to learn the same example using
19 equivalence and 30 membership queries.

