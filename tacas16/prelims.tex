\section{Preliminaries}

In symbolic automata, transitions carry predicates over a decidable Boolean algebra.
%\begin{definition}[Effective Boolean Algebra]
    An \emph{effective Boolean algebra} $\mathcal{A}$
    is a tuple $(\mathfrak{D}, \Psi, \den{\_}, \bot, \top, \vee, \wedge, \neg)$ where
    $\mathfrak{D}$ is a set of \emph{domain elements};
    $\Psi$ is a set of \emph{predicates} closed under the Boolean connectives, with $\bot, \top \in \Psi$;
    $\den{\_} : \Psi \rightarrow 2^\mathfrak{D}$
    is a \emph{denotation function} such that
    \rone $\den{\bot} = \emptyset$,
    \rtwo $\den{\top} = \mathfrak{D}$,
    and \rthree for all $\varphi, \psi \in \Psi$,
        $\den{\varphi \vee \psi} = \den{\varphi} \cup \den{\psi}$,
        $\den{\varphi \wedge \psi} = \den{\varphi} \cap \den{\psi}$,
        and $\den{\neg \varphi} = \mathfrak{D} \setminus \den{\varphi}$.
%\end{definition}

%Since $\Psi$ is closed with respect to $\vee$, $\wedge$, and $\neg$,
%we can \emph{generate} $\Psi$ from an initial set $\Psi_0$ as follows:
%for all $i > 0$,
%\begin{align*}
%    \Psi_{i+1} = \Psi_i &\cup \{\neg \varphi \mid \varphi \in \Psi_i\}
%        \cup \{\varphi_1 \vee \varphi_2 \mid \varphi_1, \varphi_2 \in \Psi_i\} 
%        \cup \{\varphi_1 \wedge \varphi_2 \mid \varphi_1, \varphi_2 \in \Psi_i\}
%\end{align*}
%And finally, $\Psi = \bigcup_i \Psi_i$ for all $i \geq 0$.
%
%Consider the following examples of Boolean algebras:

\begin{example}[Equality Algebra]
    %Let $c$ be a fixed variable.
    The \emph{equality algebra} for an arbitrary set $\mathfrak{D}$
    has predicates formed from Boolean combinations of
    formulas of the form $\lambda c \ldotp c = a$ where $a \in \mathfrak{D}$.
    Formally, $\Psi$ is generated from the Boolean closure of
    $\Psi_0 = \{\varphi_a \mid a \in \mathfrak{D}\} \cup \{\bot, \top\}$
    where for all $a \in \mathfrak{D}$, $\den{\varphi_a} = \{a\}$.
    Example predicates in this algebra include the predicates
    %$c = 5 \vee c = 10$ and $\neg(c = 0)$.
    $\lambda c \ldotp c = 5 \vee c = 10$ and $\lambda c \ldotp \neg(c = 0)$.
\end{example}

\begin{example}[Interval Algebra]
\label{ex:interval}
    The finite union of left-closed right-open intervals over 
    non-negative integers (i.e. $\mathbb{N}$) also forms a Boolean algebra:
    take the Boolean closure of
    $\Psi_0 = \{\varphi_{ij} \mid i,j \in \mathbb{N} \land i < j\} \cup \{\bot, \top\}$
    where $\den{\varphi_{ij}} = [i, j)$.
    Example predicates in this algebra include those
    (written as their denotation) of the form
    $[0,5) \cup [10,15)$ or $[50,\infty)$.
\end{example}

%Now we are ready to present the formalism of symbolic finite automata.

\begin{definition}[Symbolic Finite Automata]
A \emph{symbolic finite automaton} (\SFA) $M$
is a tuple $(\mathcal{A}, Q, q_\text{init}, F, \Delta)$ where
$\mathcal{A}$ is an effective Boolean algebra, called the \emph{alphabet};
$Q$ is a finite set of states;
$q_\text{init} \in Q$ is the \emph{initial state};
$F \subseteq Q$ is the set of \emph{final states};
and $\Delta \subseteq Q \times \Psi_\mathcal{A} \times Q$
is the \emph{transition relation} consisting of
a finite set of \emph{moves} or \emph{transitions}.
\end{definition}
%
\emph{Characters} are elements of $\mathfrak{D}_\mathcal{A}$,
and \emph{words} are finite sequences of characters,
or elements of $\mathfrak{D}_\mathcal{A}^*$.
The empty word of length 0 is denoted by $\epsilon$.
A move $\rho = (q_1, \varphi, q_2) \in \Delta$,
also denoted $q_1 \xrightarrow{\varphi} q_2$,
is a transition from the \emph{source} state $q_1$
to the \emph{target} state $q_2$,
where $\varphi$ is the \emph{guard} or \emph{predicate} of the move.
A move is \emph{feasible} if its guard is satisfiable.
%
For a character $a \in \mathfrak{D}_\mathcal{A}$, 
an \emph{a-move} of $M$, denoted $q_1 \xrightarrow{a} q_2$
is a move $q_1 \xrightarrow{\varphi} q_2$
such that $a \in \den{\varphi}$.

An \SFA\ $M$ is \emph{deterministic} if, for all transitions
$(q, \varphi_1, q_1), (q, \varphi_2, q_2) \in \Delta$,
$q_1 \neq q_2 \rightarrow \den{\varphi_1 \wedge \varphi_2} = \emptyset$;
i.e., for each state $q$ and character $a$ there is at most one $a$-move
out of $q$.
An \SFA\ $M$ is \emph{complete} if, for all $q \in Q$,
$\bigvee_{(q, \varphi_i, q_i) \in \Delta} \varphi_i = \top$;
i.e., for each state $q$ and character $a$ there exists an $a$-move
out of $q$.
Throughout the paper we assume all \SFAs\ are
deterministic and complete, since
determinization and completion are always possible~\cite{dantoni14}.
An example \SFA\ is $\mathbf{M_{11}}$ in Figure~\ref{fig:example}.
This {\SFA} has 4 states and it operates over the interval algebra from Example~\ref{ex:interval}.

Given an \SFA\ $M = (\mathcal{A}, Q, q_\text{init}, F, \Delta)$
and a state $q \in Q$, 
we say a word $w = a_1 a_2 \ldots a_k$ is \emph{accepted at state $q$}
if, for $1 \leq i \leq k$, there exist moves $q_{i-1} \xrightarrow{a_i} q_i$
such that $q_0 = q$ and $q_k \in F$.
We refer to the set of words accepted at $q$ as the
\emph{language accepted at $q$}, denoted as $\mathcal{L}_q(M)$;
the \emph{language accepted by $M$} is
$\mathcal{L}(M) = \mathcal{L}_{q_\text{init}}(M)$.
%
The \SFA\ $\mathbf{M_{11}}$ in Figure~\ref{fig:example}
accepts, among others, words consisting only of numbers accepted by the predicate $[0,51)\cup[101,\infty)$
and rejects, among others, the word  $51,25$.
